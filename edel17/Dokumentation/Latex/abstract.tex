\vspace*{2cm}

\begin{center}
    \textbf{Abstract}
\end{center}

\vspace*{1cm}

\noindent DMX is a common standard in the entertainment technology industry that
allows controlling lighting fixtures (like spotlights or strobe lights
on a stage) connected over a bus. The DMX source driving the bus is
usually a mixing desk console; alternatively, lighting control software
on a computer together with a physical PC-DMX interface is often used.
The high costs of professional desk consoles and DMX interfaces limits
small associations like youth groups to less expensive interfaces that
often lack useful features.

This Bachelor's Thesis defines requirements for a PC-DMX interface in
this use case, including a price limit and advanced features like the
availability of a DMX input port to enable haptic control of software
functions. A market study is conducted, which reveals that no existing
products match all requirements, so a system design is established,
building upon the affordable single-board computer Raspberry Pi. The
software basis is provided by \emph{Open Lighting Architecture}
(\emph{OLA}), an open-source software that is able to convert different
protocols to and from DMX and process the data internally.

OLA is then extended to support DMX output through a USB-DMX adapter by
reverse engineering and implementing its protocol. DMX input directly on
Raspberry Pi's hardware is possible for the first time through the
implementation of SPI bus sampling; other tested approaches were not
successful. The validation of the finished PC-DMX interface shows
satisfactory fulfillment of the requirements. Future work to improve the
system design by making it even less expensive, more easy to use or by
adding extra features is possible.